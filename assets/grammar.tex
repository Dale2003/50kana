\subsection{数词与量词}
\subsubsection{数词}

⽇语的基数词分为汉语性数词和⽇本固有的数词两种。汉语性数词的⽤法与汉语相同,表⽰疑问时⽤“何”(なん)。

\begin{tabularx}{\textwidth}{l *{5}{>{\centering\arraybackslash}X}}
\begin{tabular}{l}1\\いち\end{tabular} &
\begin{tabular}{l}2\\に\end{tabular}&
\begin{tabular}{l}3\\さん\end{tabular}&
\begin{tabular}{l}4\\し/よん\end{tabular}&
\begin{tabular}{l}5\\ご\end{tabular}\\ \hline
\begin{tabular}{l}6\\ろく\end{tabular} &
\begin{tabular}{l}7\\なな/しち\end{tabular}&
\begin{tabular}{l}8\\はち\end{tabular}&
\begin{tabular}{l}9\\く/きゅう\end{tabular}&
\begin{tabular}{l}10\\じゅう\end{tabular}\\ \hline
\begin{tabular}{l}11\\じゅういち\end{tabular} &
\begin{tabular}{l}12\\じゅうに\end{tabular}&
\begin{tabular}{l}13\\じゅうさん\end{tabular}&
\begin{tabular}{l}14\\じゅうよん\end{tabular}&
\begin{tabular}{l}15\\じゅうご\end{tabular}\\ \hline
\begin{tabular}{l}20\\にじゅう\end{tabular} &
\begin{tabular}{l}30\\さんじゅう\end{tabular}&
\begin{tabular}{l}40\\よんじゅう\end{tabular}&
\begin{tabular}{l}50\\ごじゅう\end{tabular}&
\begin{tabular}{l}60\\ろくじゅう\end{tabular}\\ \hline
\begin{tabular}{l}100\\ひゃく\end{tabular}&
\begin{tabular}{l}200\\にひゃく\end{tabular} &
\begin{tabular}{l}300\\さんひゃく\end{tabular}&
\begin{tabular}{l}400\\よんひゃく\end{tabular}&
\begin{tabular}{l}500\\ごひゃく\end{tabular}\\ \hline
\begin{tabular}{l}1000\\せん\end{tabular}&
\begin{tabular}{l}2000\\にせん\end{tabular} &
\begin{tabular}{l}3000\\さんせん\end{tabular}&
\begin{tabular}{l}4000\\よんせん\end{tabular}&
\begin{tabular}{l}10000\\いちまん\end{tabular}&

\end{tabularx}
日本固有的数词有10个,数量大于10的时候用汉语数词表示,表示疑问用幾つ(いくつ)
\begin{tabularx}{\textwidth}{l *{5}{>{\centering\arraybackslash}X}}
\begin{tabular}{l}1つ\\ひとつ\end{tabular} &
\begin{tabular}{l}2つ\\ふたつ\end{tabular}&
\begin{tabular}{l}3つ\\みつつ\end{tabular}&
\begin{tabular}{l}4つ\\よつつ\end{tabular}&
\begin{tabular}{l}5つ\\いつつ\end{tabular}\\ \hline
\begin{tabular}{l}6つ\\むつつ\end{tabular} &
\begin{tabular}{l}7つ\\ななつ/しち\end{tabular}&
\begin{tabular}{l}8つ\\やつつ\end{tabular}&
\begin{tabular}{l}9つ\\ここのつ\end{tabular}&
\begin{tabular}{l}10\\とお\end{tabular}\\ 
\end{tabularx}

\subsubsection{量词}
\paragraph{1.本}
\begin{tabularx}{\textwidth}{l *{4}{>{\centering\arraybackslash}X}}
\begin{tabular}{l}1本\\いち\textbf{ぽ}ん\end{tabular} &
\begin{tabular}{l}2本\\にほん\end{tabular}&
\begin{tabular}{l}3本\\さん\textbf{ぽ}ん\end{tabular}&
\begin{tabular}{l}4本\\よんほん\end{tabular}\\ \hline 
\begin{tabular}{l}5本\\ごほん\end{tabular}&
\begin{tabular}{l}6本\\ろっ\textbf{ぽ}ん\end{tabular} & 
\begin{tabular}{l}7本\\ななほん\\しちほん\end{tabular}& 
\begin{tabular}{l}8本\\はっ\textbf{ぽ}ん\\はちほん\end{tabular}\\ \hline 
\begin{tabular}{l}9本\\きゅうほん\end{tabular}& 
\begin{tabular}{l}10本\\じゅっ\textbf{ぽ}ん\\じっ\textbf{ぽ}ん\end{tabular} & 
\begin{tabular}{l}11本\\じゅういちほん\end{tabular} & 
\begin{tabular}{l}12本\\じゅうにほん\end{tabular}\\ \hline 
\begin{tabular}{l}20本\\にじゅうほん\end{tabular}& 
\begin{tabular}{l}21本\\にじゅういちほん\end{tabular}& 
\begin{tabular}{l}22本\\にじゅうにほん\end{tabular}&
\begin{tabular}{l}23本\\にじゅうさん\textbf{ぼ}ん\end{tabular} \\ \hline 
\begin{tabular}{l}24本\\にじゅうよんほん\end{tabular}& 
\begin{tabular}{l}何本\\なん\textbf{ぼ}ん\end{tabular}& 
\end{tabularx}
\paragraph{2.人}
\begin{tabularx}{\textwidth}{l *{4}{>{\centering\arraybackslash}X}}
\begin{tabular}{l}1人\\ひとり\end{tabular} & 
\begin{tabular}{l}2人\\ふたり\end{tabular}& 
\begin{tabular}{l}3人\\さんにん\end{tabular}& 
\begin{tabular}{l}4人\\よにん\end{tabular}\\ \hline 
\begin{tabular}{l}5人\\ごにん\end{tabular}&
\begin{tabular}{l}6人\\ろくにん\end{tabular}&
\begin{tabular}{l}7人\\しちにん\\ななにん\end{tabular} & 
\begin{tabular}{l}8人\\はちにん\end{tabular}\\ \hline 
\begin{tabular}{l}9人\\きゅうにん\\くにん\end{tabular}& 
\begin{tabular}{l}10人\\じゅうにん\end{tabular}& 
\begin{tabular}{l}11人\\じゅういちにん\end{tabular} &
\begin{tabular}{l}12人\\じゅうににん\end{tabular} \\ \hline 
\begin{tabular}{l}20人\\にじゅうにん\end{tabular}& 
\begin{tabular}{l}30人\\さんじゅうにん\end{tabular}& 
\begin{tabular}{l}100人\\ひゃくにん\end{tabular}& 
\begin{tabular}{l}1000人\\せんにん\\いっせんにん\end{tabular}\\ \hline 
\begin{tabular}{l}10000人\\いちまんにん\end{tabular} & 
\begin{tabular}{l}何人\\なんにん\end{tabular}& 
\end{tabularx}

\paragraph{3.円}
\begin{tabularx}{\textwidth}{l *{4}{>{\centering\arraybackslash}X}}
\begin{tabular}{l}1円\\いちえん\end{tabular} & 
\begin{tabular}{l}5円\\ごえん\end{tabular}& 
\begin{tabular}{l}10円\\じゅうえん\end{tabular}& 
\begin{tabular}{l}50円\\ごじゅうえん\end{tabular}& 
\begin{tabular}{l}100円\\ひゃくえん\end{tabular}\\ \hline 
\begin{tabular}{l}500円\\ごひゃくえん\end{tabular} & 
\begin{tabular}{l}1000円\\せんえん\end{tabular}& 
\begin{tabular}{l}2000円\\にせんえん\end{tabular}& 
\begin{tabular}{l}5000円\\ごせんえん\end{tabular}& 
\begin{tabular}{l}10000円\\いちまんえん\end{tabular}\\ \hline 
\begin{tabular}{l}何円\\なんえん\end{tabular}& % 
\begin{tabular}{l} & \end{tabular}& 
\begin{tabular}{l} & \end{tabular}&
\begin{tabular}{l} & \end{tabular}&
\begin{tabular}{l} & \end{tabular}\\
\end{tabularx}
\paragraph{4.階}
\paragraph{5.時}
\paragraph{6.時間}
\paragraph{7.時半}
\paragraph{8.分}
\paragraph{9.歳}
\paragraph{10.週間}
\paragraph{11.日}
\paragraph{12.回}
\paragraph{13.月}
\paragraph{14.か月}

\subsubsection{星期}
\begin{itemize}
    \item 月曜日:げつようび
    \item 火曜日:かようび
    \item 水曜日:すいようび
    \item 木曜日:もくようび
    \item 金曜日:きんようび
    \item 土曜日:どようび
    \item 日曜日:にちようび
\end{itemize}
\subsection{こそあど系列词}
\paragraph{これ、それ、あれ、どれ}
指示代词,属于体言,可以单独使用
\begin{tabularx}{\textwidth}{l *{3}{>{\centering\arraybackslash}X}}
\rowcolor{grey!25}
\textbf{近称} & \textbf{中称} & \textbf{远称} & \textbf{疑问称} \\
 これ&それ&あれ&どれ\\
 这、这个&那、那个&那、那个&哪、哪个\\

\end{tabularx}
\begin{itemize}
    \item 近称:谈话人指与自己距离近的事物
    
    \textbf{これ}は、母の写真ではありません
    \item 中称:谈话人指与自己距离远而离谈话对方距离近的事物

    \textbf{それ}は木村さんの家族の写真ですか
    \item 远称:谈话人双方都距离所指事物远

    \textbf{あれ}は何ですか
    \item 疑问称:表示疑问或不定

    妹さんの写真は\textbf{どれ}ですか
\end{itemize}

\paragraph{ここ、そこ、あそこ、どこ}
指示代词,属于体言,用于指代场所
\begin{tabularx}{\textwidth}{l *{3}{>{\centering\arraybackslash}X}}
\rowcolor{grey!25}
\textbf{近称} & \textbf{中称} & \textbf{远称} & \textbf{疑问称} \\
 ここ&そこ&あそこ&どこ\\
 这、这里&那、那里&那、那里&哪、哪里\\

\end{tabularx}
\begin{itemize}
    \item 近称:\textbf{ここ}は私の故郷です
    \item 中称:先生の席は\textbf{そこ}ですか
    \item 远称:\textbf{あそこ}は私の大学です
    \item 疑问称:馬さんの故郷は\textbf{どこ}ですか
\end{itemize}

\paragraph{この、その、あの、どの}
连体词,只能做连体修饰语修饰后面的体言,不能够单独使用,用于指示与限定所指代的人或事物
\begin{tabularx}{\textwidth}{l *{3}{>{\centering\arraybackslash}X}}
\rowcolor{grey!25}
\textbf{近称} & \textbf{中称} & \textbf{远称} & \textbf{疑问称} \\
 この&その&あの&どの\\
 这〜&那〜&那〜&哪〜\\

\end{tabularx}
\begin{itemize}
    \item 近称:\textbf{この}犬は僕は高校時代の恋人でした
    \item 中称:\textbf{その}写真は木村さんの家族の写真ですか
    \item 远称:\textbf{あの}人は佐藤さんです
    \item 疑问称:\textbf{どこ}が馬さんの席ですか
\end{itemize}
\paragraph{こんな、そんな、あんな、どんな}
连体词,⽤来修饰体⾔,在句中做连体修饰语
\begin{tabularx}{\textwidth}{l *{3}{>{\centering\arraybackslash}X}}
\rowcolor{grey!25}
\textbf{近称} & \textbf{中称} & \textbf{远称} & \textbf{疑问称} \\
 こんな&そんな&あんな&どんな\\
 这样的&那样的&那样的&哪样的\\

\end{tabularx}
\begin{itemize}
    \item \textbf{こんな}ことは初めてです。
    \item \textbf{そんな}⼈はあまり好きではありません。
    \item \textbf{あんな}事件は知りませんでした。
    \item \textbf{どんな}のを買いましたか。
\end{itemize}
\paragraph{こう、そう、ああ、どう}
在句中做连⽤修饰语,⽤于修饰⽤⾔。
\begin{tabularx}{\textwidth}{l *{3}{>{\centering\arraybackslash}X}}
\rowcolor{grey!25}
\textbf{近称} & \textbf{中称} & \textbf{远称} & \textbf{疑问称} \\
 こう&そう&ああ&どう\\
 这样、这么&那样、那么&那样、那么&怎样、如何\\

\end{tabularx}
\begin{itemize}
    \item 正座は\textbf{こう}するんですか。
    \item \textbf{そう}してください。
    \item 先⽣のお名前は\textbf{どう}書きますか。
\end{itemize}
“ああ”多⽤固定的説法、如:“ああでもない、こうでもない。”意为:“这样做也不是,那样做也不是
\paragraph{こちら、そちら、あちら、どちら}
指⽰代词,属于体⾔,⽤来指代⽅位或场所,在句中可以做主语或谓语。另外,它还⽤于指代⼈。
\begin{tabularx}{\textwidth}{l *{3}{>{\centering\arraybackslash}X}}
\rowcolor{grey!25}
\textbf{近称} & \textbf{中称} & \textbf{远称} & \textbf{疑问称} \\
 こちら&そちら&あちら&どちら\\
 这边&那边&那边&哪边\\

\end{tabularx}
\begin{itemize}
    \item \textbf{こちら}は中国料理の屋台です。
    \item \textbf{そちら}は⾺さんです。
    \item \textbf{あちら}は千曲川です。
    \item ⽊村さんは今\textbf{どちら}にいますか。
\end{itemize}
\subsection{称谓}
\begin{tabularx}{\textwidth}{l *{6}{>{\centering\arraybackslash}X}}
\rowcolor{grey!25}
& \textbf{父亲} & \textbf{母亲} & \textbf{哥哥}& \textbf{姐姐}& \textbf{弟弟}& \textbf{妹妹} \\
\textbf{称呼他人家庭成员} &お父さん&お母さん&お兄さん&お姉さん&弟さん&妹さん\\
\textbf{直呼自己家庭成员}&お父さん&お母さん&(お)兄さん&(お)姉さん&直呼名&直呼名\\
\textbf{谈论自己家庭成员}&父&母&兄&姉&弟&妹\\

\end{tabularx}

\subsection{时间的表达方式}
\begin{tabularx}{\textwidth}{l *{4}{>{\centering\arraybackslash}X}}
\rowcolor{grey!25}
\textbf{上上} & \textbf{上} & \textbf{当今}& \textbf{下}& \textbf{下下} \\
\begin{tabular}{l}おととし\\一昨年\end{tabular} &
\begin{tabular}{l}きょねん\\去年\end{tabular}&
\begin{tabular}{l}ことし\\今年\end{tabular}&
\begin{tabular}{l}らいねん\\来年\end{tabular}&
\begin{tabular}{l}さらいねん\\再来年\end{tabular}\\ \hline
\begin{tabular}{l}せんせんげつ\\先々月\end{tabular} &
\begin{tabular}{l}せんげつ\\先月\end{tabular}&
\begin{tabular}{l}こんげつ\\今月\end{tabular}&
\begin{tabular}{l}らいげつ\\来月\end{tabular}&
\begin{tabular}{l}さらいげつ\\再来月\end{tabular}\\ \hline
\begin{tabular}{l}おととい\\一昨日\end{tabular} &
\begin{tabular}{l}きのう\\昨日\end{tabular}&
\begin{tabular}{l}きょう\\今日\end{tabular}&
\begin{tabular}{l}あした\\明日\end{tabular}&
\begin{tabular}{l}あさって\\明後日\end{tabular}\\ \hline
\begin{tabular}{l}せんせんしゅう\\先々週\end{tabular} &
\begin{tabular}{l}せんしゅう\\先週\end{tabular}&
\begin{tabular}{l}こんしゅう\\今週\end{tabular}&
\begin{tabular}{l}らいしゅう\\来週\end{tabular}&
\begin{tabular}{l}さらいしゅう\\再来週\end{tabular}\\ \hline
\begin{tabular}{l}あさ\\朝\end{tabular} &
\begin{tabular}{l}ごぜん\\午前\end{tabular}&
\begin{tabular}{l}ひる\\昼\end{tabular}&
\begin{tabular}{l}ごご\\午後\end{tabular}&
\begin{tabular}{l}よる\\夜\end{tabular}\\ 

\end{tabularx}

\section{语法}
\subsection{句式}
11,12课,文型暂缺
\subsubsection{判断句}
\paragraph{1. 名词句 です}

\begin{itemize}
    \item A は B です。—— “A 是 B。”
\end{itemize}

\textbf{例:}  
はい、大学生です。  
(是的,我是大学生。)

\paragraph{2. 名词否定句 ではありません}

\begin{itemize}
    \item A は B ではありません。—— “A 不是 B。”
\end{itemize}

\textbf{例:}  
いいえ、カラオケではありません。  
(不,不是在卡拉 OK。)


\paragraph{3. 名词句过去式 でした/ではありませんでした}

\begin{itemize}
    \item A は B でした。——“A 过去是 B。”
    \item A は B ではありませんでした。——“A 过去不是 B。”
\end{itemize}

\textbf{例:}  
学生でした。  
(以前是学生。)

\paragraph{4. 疑问句 ですか}

\begin{itemize}
    \item A は B ですか。——“A 是 B 吗?”
    \item 回答一般用:はい/いいえ
\end{itemize}

\textbf{例:}  
大学生でしたか。  
(以前是大学生吗?)

\subsubsection{描写句:形容词性陈述句}

\paragraph{1. は 形容词1 い です}

这是形容词 1(い形容词)的陈述句敬体肯定形式,用于表达性质、状态。

例:
\begin{itemize}
    \item 発音は難しいです。
    \item それはよい本です。
    \item その映画はおもしろいです。
\end{itemize}

\paragraph{2. はい形容词1 くないです/ありません}

形容词 1 的否定形式,“い”变成“くない”。  
两种表达均可:“くないです”(更口语)/“くありません”(更正式)。

例:
\begin{itemize}
    \item この道は広くないです。
    \item 部屋はあまり暑くありません。
\end{itemize}

\paragraph{3. はい形容词1 かったです}

形容词 1 的过去式肯定形式,“い”变“かった”。

例:
\begin{itemize}
    \item 昨日は暑かったです。
    \item 勉強はおもしろかったです。
    \item 昨日は涼しかったです。
\end{itemize}


\paragraph{4. はい形容词1 くなかったです/ありませんでした}

形容词 1 的过去否定形式,表示过去“不……”。

例:
\begin{itemize}
    \item 去年の冬は寒くなかったです/ありませんでした。
    \item 昨日は暑くなかったです/ありませんでした。
    \item 昨日は涼しくなかったです/ありませんでした。
\end{itemize}

\paragraph{5. あまり \quad くないです/ありません}

副词「あまり」用于否定句,表示“不太……”、“不很……”。

例:
\begin{itemize}
    \item 日本語の発音はあまり難しくないです/ありません。
    \item その山はあまり高くないです/ありません。
    \item 今日はあまり寒くないです/ありません。
\end{itemize}

\paragraph{6. は\quad ほとんど \quad くないです/ありません}

「ほとんど」表示“几乎不……”。

例:
\begin{itemize}
    \item 北京は東京ほど寒くないです/ありません。
    \item 日本語の発音は中国語ほど難しくないです/ありません。
    \item 佐藤さんは馬さんほど高くないです/ありません。
\end{itemize}

\paragraph{7. ~は 形容词2(な形容词) です/ではありません}

これは形容词2(な形容词)を述语とする敬体基本句型。  
「です」结句为肯定句,「ではありません」为否定句。

例:
\begin{itemize}
    \item ここの紅葉は有名です/ではありません。
    \item 故郷の交通は便利です/ではありません。
\end{itemize}


\paragraph{8. ~は 形容词2 でした/ではありませんでした}

な形容词的过去式:「でした」表示过去肯定,「ではありませんでした」表示过去否定。

例:
\begin{itemize}
    \item あの川はきれいでした/ではありませんでした。
    \item ここはにぎやかでした/ではありませんでした。
    \item ここのいちごは有名でした/ではありませんでした。
\end{itemize}

\paragraph{9. ~は~が 好きです/嫌いです}

A は B が 好きです/嫌いです。  
“は”表示主题,“が”表示喜恶对象,“好き/嫌い”作为整个述语部分。

例:
\begin{itemize}
    \item 私はいちごが好きです。
    \item 子どもの時は、にぎやかなところが好きでした。
    \item 私は夏が嫌いです。
\end{itemize}

\subsubsection{存在句}

\paragraph{1. ~は~に あります/います}
这是存在句敬体表达形式的⾮过去时肯定形式,表⽰事物、⼈或动
物存在于某处。
⼀般译为“~在~”。表⽰事物存在⼦某处时⽤“
す”表⽰⼈或物存在某⽤“います”。它的否定形式是把“ま
す”換成“ません”。
\begin{itemize}
    \item お菓子やジュースなどは生協にあります。
    \item ボールペンは店にありません。
    \item 学生はここにいます。
    \item 父は会社にはいません。
\end{itemize}

---

\paragraph{2. ~には ~があります/います}
这是存在句敬体表达⽅式的另⼀种表⽰⾮过去时肯定形式的句式。
表⽰在某处有某物、⼈或动物。
⼀般译为“在~有~”。表⽰在某处有某物時⽤“あります”;表⽰在某処有某⼈、物⽤“います”。它的否定形式是把“ます”変成“ません”。 
\begin{itemize}
    \item 生協にはお菓子やジュースなどがあります。
    \item 店には野菜がありません。
    \item そこには学生がいません。
\end{itemize}

\paragraph{3. ~のほかに,~とか,~も あります/います}
意思“除~之外”。意为“除~之外、有~和~”。另外,正常以“~のほかに、~と~もあります/います”的形式出现,可以译为“除~之外,还有~和~”。

\begin{itemize}
    \item 「~のほかに」= “~のほかにも/~以外にも”。
    \item 「~とか」=例示,“~や~など”。
    \item 「~も」=包含的强调。
\end{itemize}

例:
\begin{itemize}
    \item 本部キャンパスには教育学部のほかに,人文学部と体育学部があります。
    \item 部屋には佐藤さんのほかに,田中さんと木村さんがいます。
    \item 教室には留学生のほかに,中国の学生もいます。
\end{itemize}

\paragraph{4. ~に~や~などがあります}

「~や~など」用于列举事物,“~や~など(が)ある”=“在~有~、也有~等”。

例:
\begin{itemize}
    \item 大学には文学部や体育学部などがあります。
    \item 売店に文房具や本などがあります。
    \item 部屋には電話や机などがあります。
\end{itemize}


\paragraph{5. ~にしか ありません/いません}

「しか」+否定=“只……而已”“只有……”。  
表示范围限定,必须与否定形式(ありません/いません)搭配。

例:
\begin{itemize}
    \item 部屋に机と電話しかありません。
    \item 教室に留学生しかいません。
    \item 庭に桜の木しかありません。
\end{itemize}

\paragraph{6. ~は~が~}

主谓结构句型,“は”提示全句主题,“が”提示主要叙述对象(句子的主语部分)。  
谓语部分由「~が~」组成。

例:
\begin{itemize}
    \item 大学の売店は店員が少ないです。
    \item 日本語は発音が難しくありません。
    \item 冬は風が冷たいです。
\end{itemize}
\subsubsection{叙述句}
以动词 (表示存在的动词 “あります” “います” 除外) 做谓语的句子叫叙述句,它用于叙述人或事物的行为、动作、状态、作用等。
\paragraph{~ながら、~ます}
“ながら”接在动词第⼀连⽤形的后⾯,表⽰同⼀主体同时进⾏的两个动作。
⼀般意为“⼀边~,⼀边~”。例如:
\begin{itemize}
    \item 留学⽣たちはギョーザを⾷べ\textbf{ながら}、⼤学のことを\textbf{話しています}
    \item わたしはいつも⾳楽を聞き\textbf{ながら}、部屋の掃除を\textbf{します}
    \item ⾺さんはご飯を⾷べ\textbf{ながら}、テレビ\textbf{を⾒ていま}
\end{itemize}
\begin{itemize}
    \item 馬さんは嵐山で写真を撮りました
    \item わたしは毎日 7 時に起きます
\end{itemize}
\subsection{形式名词}
\subsubsection{ほう}
\subsubsection{の}
属于体⾔,它接在⽤⾔连体形后⾯,起语法作⽤,使其名词
化。可以在句中做主语、宾语。
\begin{itemize}
    \item わたしは⾚い\textbf{の}を選びました。
    \item あのきれいな\textbf{の}は何ですか。
    \item ⼤きい\textbf{の}はいくらですか。
\end{itemize}
\subsection{助词}


\subsubsection{提示助词}
\paragraph{1.は}
\subparagraph{1}
接在体言之后,读作わ,表示对于主题,话题的提示或强调,意为“是”
\begin{itemize}
    \item 馬さん\textbf{は}大学生です
    \item 父\textbf{は}会社員です
    \item 妹\textbf{は}高校2年生です
\end{itemize}
\subparagraph{2}
接在体言后面,用于对比出现在两个分句中的主语
\begin{itemize}
    \item 日本語の発音\textbf{は}難しくないですか、助詞の使い方\textbf{は}難しいです
    \item 子どもの時\textbf{は}、賑やかな都会が好きでしたが、今\textbf{は}、静かな郊外好きです
    \item 夏\textbf{は}暑いですが、冬\textbf{は}寒いです
\end{itemize}
上述句子中が的用法参见接续助词
\subparagraph{3}
接体⾔后⾯,可以起到宾语前置的作⽤。即把宾语提前到句
⾸作为主题,表⽰强调。
\begin{itemize}
    \item 写真\textbf{は}撮りませんでした。
    \item 昼⾷\textbf{は}どこでとりましたか。
    \item 紅葉\textbf{は}⾒ませんでした
\end{itemize}
\paragraph{2.も}
\subparagraph{1}
表示对前者同类事物的重复,意为“也”
\begin{itemize}
    \item 父は医者です。母\textbf{も}医者です。
    \item マリーさんの趣味はカラオケです。私の趣味\textbf{も}カラオケです。
\end{itemize}
\subparagraph{2}
\paragraph{3.でも}

\subsubsection{格助词}
\paragraph{1.の}
\subparagraph{1}
接在体言之后,做连体修饰语(定语),表领属、属性等,意为“的”
\begin{itemize}
    \item 僕\textbf{の}父は医者です。
    \item 父は高校\textbf{の}教師です。
    \item これは木村さん\textbf{の}家族\textbf{の}写真ですか。
\end{itemize}
\subparagraph{2}
\paragraph{2.が}
\subparagraph{1}
接在体言后面,在句中做主语
\begin{itemize}
    \item 秋は涼しくて、気持ち\textbf{が}いいですね
    \item 風\textbf{が}冷たいです
    \item この方\textbf{が}中村先生です
\end{itemize}
\subparagraph{2}
接在体言后,表示好恶的对象
\subparagraph{3}
接在体言后面,表示主语。疑问词作主语时一定要接“が”,相应的回答也要用“が”
\begin{itemize}
    \item A:どれ\textbf{が}馬さんのボールペンですか

    B:どれ\textbf{が}馬さんのボールペンです
    
    \item A:どれ\textbf{が}国際交流センターですか

    B:どれ\textbf{が}国際交流センターです
\end{itemize}

\paragraph{3.より}
接在体言后面,表示比较的对象或基准,在句中作补语,意为“比~、与~相比”
\begin{itemize}
    \item 馬さんは佐藤さん\textbf{より}高いです
    \item 中国語の発音は日本語\textbf{より}難しいです
    \item 今日の勉強は昨日\textbf{より}面白いです
\end{itemize}
\paragraph{4.に}
\subparagraph{1}
接在体言后面,表示人、动物或事物存在的位置、场所,在句中作补语,意为“zai~”
\begin{itemize}
    \item 売店\textbf{に}店員か2人います
    \item そこ\textbf{に}新しい建物かあります
    \item 学生は教室\textbf{に}います
\end{itemize}
\subparagraph{2}
接在体言后面,表示比较、评价的基准,在句中作补语,意为“离~”、“对~”
\begin{itemize}
    \item ここは彼の家\textbf{に}近いです
    \item 木村さんの家は大学\textbf{に}遠いです
    \item この辞書は留学生\textbf{に}いいです
\end{itemize}
\subparagraph{3}
接在有关时间的体⾔后⾯,表⽰动作、⾏为等进⾏的时间、
时点,在句中做状语。
⼀般译为“在~”,或不译出。
\begin{itemize}
    \item 会議は1時\textbf{に}始まります。
    \item 授業は6時半\textbf{に}終わります。
    \item 何時\textbf{に}⾏きますか。
\end{itemize}
\subparagraph{4} 
接在名词后⾯、与“なる”搭配使⽤,表⽰事物展、変化的状态、结果。

~になります/~くなります

“なる”表⽰事物袋展和変化的状、結果。它接在体⾔和形容詞2
后⾯时前⾯加“に”;接形容词1后⾯时,要先把形容词1变成第⼀连⽤
(く形)。⼀般译为“成为~”“变成~”。
\begin{itemize}
    \item ⾺さんは今⽇で20歳\textbf{になります}ね。
    \item 故郷の交通は便利\textbf{になりました}ね。
    \item これから寒\textbf{くなります}よ
\end{itemize}
\subparagraph{5}
接在动词第⼀连⽤形后⾯,后续“行く”“来る”“帰る”等移动动词,表⽰⾏为、动作的⽬的等。
\begin{itemize}
    \item 佐藤さんはセーターを届け\textbf{に}来ました。
    \item ⽊村さんは本をもらい\textbf{に}⾏きました。
    \item 彼⼥は⽣協へジュースを買い\textbf{に}⾏きます
\end{itemize}
\subparagraph{6}
接在体⾔后⾯,表⽰动作、⾏为的⽅向。
\begin{itemize}
    \item あした、東京\textbf{に}⾏きます。
    \item 母と⼀緒\textbf{に}家に帰りま
\end{itemize}
\subparagraph{7}
接在体⾔后⾯,表⽰动作、作⽤的归着点。
\begin{itemize}
    \item わさびを醤油の中\textbf{に}⼊れてください。
    \item ⽊村さんは家族の写真を本の中\textbf{に}⼊れました。
\end{itemize}
\subparagraph{8}
接在体⾔后⾯,表⽰先后两者的并列、添加关系。可以译
⽅“和”“及”。:
\begin{itemize}
    \item わたし\textbf{に}はいい友だちが3⼈います。⾺さん、さんに⽊村さんです。
    \item ⽔ギョーザ\textbf{に}お酒は最⾼です。
    \item 佐藤さんからセーター\textbf{に}ケーキをもらいました。
\end{itemize}
\subparagraph{9}
接在体⾔后⾯,表⽰事物、物体留在或附着在某处。:
\begin{itemize}
    \item ここ\textbf{に}名前を書いてください。
    \item 本を机の上\textbf{に}置きました。
    \item ⽊村さんは飯⽥橋\textbf{に}住んでいます
\end{itemize}
\paragraph{5.から}
接在体言后面,表示时间、空间的起点,在句中作补语,意为“从~起”、“离~”、“由~”
\begin{itemize}
    \item 大学\textbf{から}30分ぐらいです
    \item 駅\textbf{から}遠いです
    \item 生協は何時\textbf{から}ですか
\end{itemize}
\paragraph{6.と}
接在体⾔后⾯,表⽰共同进⾏某动作、⾏为的对象,在句中做补语。
⼀般译为“和~”“与~”。
\begin{itemize}
    \item 母\textbf{と}⽥舎へ⾏きました
    \item 同級⽣たち\textbf{と}嵐⼭へ⾏きました
    \item 妹\textbf{と}写真を撮りまし
\end{itemize}
\paragraph{7.へ}
接在有关处所的体⾔后,表⽰移动的⽅向、归着点,在句中做补语。
⼀般译为“到~”“往~”。
\begin{itemize}
    \item 彼は北京\textbf{へ}⾏きました。
    \item ⼤きい教室\textbf{へ}⾏きます。
    \item 来⽉、⽇本\textbf{へ}⾏きます
\end{itemize}
\paragraph{8.で}
\subparagraph{1}
接在体⾔后⾯,表⽰⾏、动作使⽤的⼯具、⼿段或⽅法,
在句中做补语。
⼀般译为“⽤~”“以~”“乘~”等。
\begin{itemize}
    \item 今⽇は⾃転⾞\textbf{で}来ました。
    \item その情報はテレビ\textbf{で}⾒ました。
    \item 毎⽇,バス\textbf{で}会社へ⾏きます。
\end{itemize}
\subparagraph{2}
接有关处所的体⾔后⾯,表⽰⾏为、动作的场所,在句中做地点状语。
⼀般译为“在~”。
\begin{itemize}
    \item 嵐⼭\textbf{で}記念撮影をしました。
    \item 昼⾷は⼤学\textbf{で}取りました。
    \item 中村さんの部屋\textbf{で}テレビを⾒ました。
\end{itemize}
\subparagraph{3}
多接在表⽰数量的体⾔后⾯,表⽰对其数量的限定,在句⼦中做补语。可以译为“总共”等。
\begin{itemize}
    \item あと1週間\textbf{で}⾺さんの誕⽣⽇ですね。
    \item お客さん、6本\textbf{で}よろしいですか。
    \item ボールペンとジュース\textbf{で}230円です。
\end{itemize}
\subparagraph{4}
接在表⽰范围的体⾔后⾯,⽤来限定范围,在句中做补语。
⼀般译为“在~”等。
\begin{itemize}
    \item ⽇本\textbf{で}初めての誕⽣⽇です。
    \item そこは北京\textbf{で}⼀番にぎやかなところです。
    \item この川は中国\textbf{で}とても有名です。
\end{itemize}
\subparagraph{5}
接在体⾔后⾯,表⽰原因、理由,在句中做补语。可以译
为“由于~⽽~”“因为~所以~”。
\begin{itemize}
    \item クラブ活動\textbf{で}鎌倉へ⾏くんです。
    \item ⾬\textbf{で}⾏きませんでした。
    \item 彼⼥は観光\textbf{で}北京に来ました。
\end{itemize}
\paragraph{9.を}
\subparagraph{1}
接在体⾔后⾯,在句中做宾语,表⽰动作的对象。⼀般译为“把~”,或不译出。
\begin{itemize}
    \item 嵐⼭で写真\textbf{を}撮りました。
    \item わたしはよく電⾞\textbf{を}利⽤します。
    \item 周恩来の詩の記念碑\textbf{を}⾒ました。
\end{itemize}
\subparagraph{2}
接在表⽰场所的体⾔后⾯,在句中做补语。它与具有离开、移动等意义的⾃动词搭配使⽤,表⽰离开或移动的场所。例如:
\begin{itemize}
    \item 彼⼥は6時に家\textbf{を}出ました。
    \item 兄は今年の7⽉に⼤学\textbf{を}卒業しました。
    \item 彼らはいろいろな国の料理の屋台\textbf{を}回っていま
\end{itemize}
\paragraph{10.まで}
接在体⾔后⾯,表⽰到的时间、空间的終点,在句中做补語。
⼀般译为“到~”。
\begin{itemize}
    \item 彼の家から⼤学\textbf{まで}20分です。
    \item 1時から3時\textbf{まで}授業があります
    \item 売店は何時\textbf{まで}ですか
\end{itemize}
“~から~まで”⼀般意为“以~到~”。接在体⾔后⾯、分別表⽰时间、空间的起点和終点。在句中作补语。两者均可以单独使⽤
\begin{itemize}
    \item 授業は8時\textbf{から}12時\textbf{まで}です。
    \item 学校\textbf{から}家\textbf{まで}電⾞で10分です。
    \item トムさんの家は駅\textbf{から}遠いですか。
    \item 今⽇の授業は何時\textbf{まで}ですか
\end{itemize}
\subsubsection{并列助词}
\paragraph{1.と}
连接两个以上的体言,表示并列,意为“和”
\begin{itemize}
    \item 私は3人家族です。父、母\textbf{と}私です。
    \item 木村さん\textbf{と}トムさんは大学生です。
    \item トムさん\textbf{と}馬さんは会社員ではありません。
\end{itemize}

\paragraph{2.や}
接在体言后面,用于列举两个以上的事物,暗示言外之意还有其他,常与副助词 “など” 搭配使用。一般译为“~啦~啦”、“~呀~呀”。

\begin{itemize}
    \item 売店には、本\textbf{や}文房具などがあります。
    \item 机の上には、本\textbf{や}雑誌などがあります。
    \item そこのいちご\textbf{や}桃などは有名です。
\end{itemize}
\subsubsection{终助词}
\paragraph{1.が}
接在句尾,构成疑问句式,意为“吗?”

在日文疑问句中一般不使用问号,而是使用句号。

\paragraph{2.ね}
用于句尾,表示感叹,或者向听话人确认自己所说的内容以及征得听话人的赞同,意为“呀、啊、吧”

\begin{itemize}
    \item いいお天気です\textbf{ね}
    \item 空が青くて高いです\textbf{ね}
    \item 今日は涼しいです\textbf{ね}
\end{itemize}
\paragraph{3.よ}
用于句尾,表示加强语气,促使听话人了解或者接受自己的意见,意为“呀、哟、啊”

\begin{itemize}
    \item 北京も秋はいい季節です\textbf{よ}
    \item 助詞の使い方は面白いです\textbf{よ}
    \item 秋は涼しくて、気持ちがいいです\textbf{よ}
\end{itemize}
\subsubsection{接续助词}
\paragraph{1.が}
\subparagraph{1}
接在前一个句子末尾,用于连接前后两个句子,表示两者之间的转折关系,意为“而、可是、不过”
\begin{itemize}
    \item 寒いです\textbf{が}、北京ほど寒くありません
    \item 助詞の使い方は難しいです\textbf{が}、発音は難しくないです
    \item 去年は暖冬でした\textbf{が}、今年は寒いですね
\end{itemize}
\subparagraph{2}
接在⽤⾔终⽌形后⾯,以此来结束句⼦,做终助词使⽤。表⽰委婉的语⽓,暗⽰后续的相反的内容。
\begin{itemize}
    \item 中国ではふつう、⿂は⽣のままでは⾷べません\textbf{が}。
    \item 今⽇は忙しくていっしょに⾏きません\textbf{が}。
    \item この助詞の使い⽅は難しくて、僕には分かりません\textbf{が}
\end{itemize}
\paragraph{2.ので}
接在连体形后⾯,⽤于连接两个句⼦,表⽰前者与后者之间存在的客观的因果关系。接名词时⽤“の”的形式。
⼀般译为“因为~所以~”。
\begin{itemize}
    \item 天気が悪い\textbf{ので}、嵐⼭へ⾏きませんでした。
    \item 佐藤さんが⾏く\textbf{ので}、わたしも⾏きます。
    \item 紅葉がきれいな\textbf{ので}、写真を撮りました。
    \item 今⽇は⽇曜⽇な\textbf{ので}、⼈が多いです
\end{itemize}
\paragraph{3.から}
接在“体⾔+です”和⽤⾔简体形式后⾯,连接两个分句。表⽰说话⼈主观认为的原因和理由。
⼀般译为“因为~,所以~”。有时也以“から”结句,这种情况也可以视为终助词的⽤法。
\begin{itemize}
    \item あしたは⼟曜⽇です\textbf{から}、観光客でいっぱいになるでしょう。
    \item ここは静かだ\textbf{から}、勉強にいいです。
    \item 午後は講演がある\textbf{から}、どこへも⾏きません。
    \item 彼⼥も⼀緒に⾏くでしょう。あしたは⽇曜⽇です\textbf{から}。
\end{itemize}
\paragraph{4.けれど}
接在“体⾔+です”和以⽤⾔结句的句⼦后⾯,连接两个分句。表⽰
两者之同的折系。它也可以説成“けれども”、“いけど”

“けれど”、“けど”多⽤于口語,“けれども”既可以⽤于口語,也可以⽤书⾯語。
⼀般意为“而”“可是”“不过”等。
\begin{itemize}
    \item いい機会です\textbf{けれど}、クラブ活動があるので、残念です。
    \item 東京も寒い\textbf{けれど}、北京ほど寒くありません。
    \item 観光客の中には外国⼈もいる\textbf{けれど}、⽇本⼈が多いです。
\end{itemize}
\paragraph{5.て}
接在动词之后,构成动词第⼆连⽤形(也称“て形”)。在句
⼦中⽤于连⽤修饰语,可以表⽰如下意思
\subparagraph{1}
表⽰⾏为的⽅法和⼿段。
\begin{itemize}
    \item 醤油をつけ\textbf{て}⾷べてください。
    \item このテキストを使っ\textbf{て}⽇本語を勉強します。
    \item 朝早く起き\textbf{て}、部屋の⼤掃除をします。
\end{itemize}
\subparagraph{2}
表⽰原因和理由。
\begin{itemize}
    \item こんなにたくさんのお湯を使っ\textbf{て}、もったいなくないですか。
    \item たくさんのお客さんが来\textbf{て}、とてもにぎやかです。
    \item 約束の時間に遅れ\textbf{て}、すみません。
\end{itemize}
\subparagraph{3}
接在动词后⾯,构成动词第⼆连⽤形(“て形”)。表⽰动作、⾏为的先后顺序。例如:
\begin{itemize}
    \item ほかの屋台を回ってから、もう⼀度ここに戻っ\textbf{て}、⽔ギョーザを
    \item ⾷べましょう。
    \item 朝7時に家を出\textbf{て}、電⾞に乗りました。
    \item 今⽇、疲れたので、お⾵呂に⼊っ\textbf{て}すぐ寝まし
\end{itemize}
\subparagraph{}
\paragraph{6.ながら}
接在动词第⼀连⽤形后⾯、与接“ます”的⽅法相同,
表⽰同⼀主体同时进⾏的两个动作。(详见“⽂型”)
\paragraph{7.たり}
\paragraph{8.たら}

\subsubsection{判断助动词}
\paragraph{1.です}
接体验做判断句位于,表示肯定,意为“是”

根据词尾的活用变化可以构成判断句的否定形式或者过去式。

\paragraph{2.で}
是です的连用形,表示中顿,用于并列复句
\begin{itemize}
    \item 父は医者\textbf{で}、母は教師です
    \item 木村さんは大学生\textbf{で}、妹さんは高校2年生です
    \item 私の故郷は北京の郊外\textbf{で}、静かな所です
\end{itemize}
\subsubsection{副助词}
\paragraph{1.ほど}
接在体言后面,表示比较的基准等,与否定形式搭配使用,意为“不像~那样”
\paragraph{2.など}
接在体言后,表示概括所列举的同类事物,暗示同类中还有其他。一般译为 “~什么的” 或 “~之类”,常与并列助词“や”一起使用
\paragraph{3.しか}
接在有关数量、程度、范围等的体言后面,表示强调、限定。它与谓语的否定形式搭配使用,但表示肯定意义。意为 “仅仅~” 或 “只有~”。
\paragraph{4.か}
接在疑问词后面,表示不定或不确切的推断。

\begin{itemize}
    \item そのほかに、何\textbf{か}必要なものがありますか。
    \item そこに誰\textbf{か}いますか。
    \item 本部キャンパスに学部\textbf{が}いくつありますか。
\end{itemize}
\paragraph{5.ぐらい/くらい}
接在体言后面,表示概数。意为 “大概~”、“大约~” 或 “~左右”。

\begin{itemize}
    \item 学生はどの\textbf{ぐらい}いますか。
    \item 学生は 6000 人\textbf{ぐらい}います。
    \item 駅から 30 分\textbf{ぐらい}です。
\end{itemize}
\paragraph{6.だけ}
接在体⾔后⾯,表⽰对数量和范围的限定。
⼀般可以译为“只有~”、“仅仅~”。
\begin{itemize}
    \item 6本\textbf{だけ}でよろしいですか。
    \item 留学⽣\textbf{だけ}無料です。
    \item 今度\textbf{だけ}お⾦をあげます
\end{itemize}
\subsubsection{助词的重叠}
\paragraph{1.には}
是助词 に 和 は 的重叠,接在体言后面。表示强调,起加强语气的作用。
意为 “在~”。
\begin{itemize}
    \item わたしたちの大学\textbf{には}留学生もいます。
    \item 生協\textbf{には}何がありますか。
    \item 机の上\textbf{には}本や雑誌などがあります。
\end{itemize}
\paragraph{2.へも}
助词“へ”和“も”的重叠,“へ”表⽰⾏为、动作的⽅
向;“も”表⽰强调。与谓语的否定形式搭配使⽤,表⽰全⾯否定。可以
译为“(连)~也不(没)~”“(连)~都不(没)~”。
\begin{itemize}
    \item ⽇曜⽇はどこ\textbf{へも}⾏きません。
    \item 彼は忙しくて、故郷\textbf{へも}帰りませんでした。
    \item その⽇、忙しかったので、鎌倉\textbf{へも}⾏きませんでした。
\end{itemize}
\paragraph{3.では}
格助词“で”和提⽰助词“は”的重叠,接在体⾔后⾯,
提⽰主题,表⽰⼀定的范围。
⼀般可以译为“在~”。
\begin{itemize}
    \item 中国\textbf{では}ふつうシャワーだけです。
    \item 湯船の中\textbf{では}せっけんを使ってもいいですか。
    \item 佐藤さんのお宅\textbf{では}いつもお母さんの⼿料理です
\end{itemize}
\paragraph{4.でも}


\subsection{形容词}

\subsubsection{形容词1}
\paragraph{1.词典形}
基本形、原形。由词干和词尾组成。以“い”做词尾。

各种活用变化都在“い”上进行,“い”前面的部分为词干,不发生活用变化

\paragraph{2.连体形}
形态上与词典形完全相同,用于修饰体言,做连体修饰语(定语)

\paragraph{3.连用形}
\subparagraph{第一连用形}
“く”形,把词尾的“い”换成“く”,一般后接ないです或ありません表示否定

\begin{itemize}
    \item 暑い \quad 暑く \quad 暑\textbf{くないです/ありません}
    \item 涼しい \quad 涼しく \quad 涼し\textbf{くないです/ありません}
    \item 冷たい \quad 冷たく \quad 冷た\textbf{くないです/ありません}
\end{itemize}

\subparagraph{第二连用形}
“て”形,把词尾的“い”换成“く”,后接“て”。表并列、轻微的原因。意为“而”
“因为~所以~”

\begin{itemize}
    \item 空が青く\textbf{て}高いですね(并列)
    \item 秋は涼しく\textbf{て}、気持ちがいいです(原因)
\end{itemize}

\paragraph{4.终止形}
终止形是用言、助动词活用形式之一,形态上与词典形完全相同,用于做谓语并结句。敬体形式是由其变化而来。

敬体非过去时肯定和否定形式一般多用“第一连用形+ないです”的形式
\begin{tabularx}{\textwidth}{l *{2}{>{\centering\arraybackslash}X}}
\rowcolor{grey!25}
\textbf{词典形} & \textbf{简体非过去时肯定和否定形式} & \textbf{敬体非过去时肯定和否定形式} \\
 寒い&\begin{tabular}{l}肯定:寒い\\否定:寒くない\end{tabular} &\begin{tabular}{l}肯定:寒いです\\否定:\begin{tabular}{l}寒くないです\\寒くありません\end{tabular}\end{tabular} \\
 涼しい&\begin{tabular}{l}肯定:涼しい\\否定:涼しくない\end{tabular} &\begin{tabular}{l}肯定:涼しいです\\否定:\begin{tabular}{l}涼しくないです\\涼しくありません\end{tabular}\end{tabular} \\
 いい&\begin{tabular}{l}肯定:いい\\否定:よくない\end{tabular} &\begin{tabular}{l}肯定:いいです\\否定:\begin{tabular}{l}よくないです\\よくありません\end{tabular}\end{tabular} \\
\end{tabularx}


\begin{tabularx}{\textwidth}{l *{2}{>{\centering\arraybackslash}X}}
\rowcolor{grey!25}
\textbf{词典形} & \textbf{简体过去时肯定和否定形式} & \textbf{敬体过去时肯定和否定形式} \\
 寒い&\begin{tabular}{l}肯定:寒かった\\否定:寒くなかった\end{tabular} &\begin{tabular}{l}肯定:寒あったです\\否定:\begin{tabular}{l}寒くなかったです\\寒くありませんでした\end{tabular} \end{tabular} \\
 涼しい&\begin{tabular}{l}肯定:涼しかった\\否定:涼しくなかった\end{tabular} &\begin{tabular}{l}肯定:涼しかったです\\否定:\begin{tabular}{l}涼しくなかったです\\涼しくありませんでした\end{tabular}\end{tabular} \\
 いい&\begin{tabular}{l}肯定:よかった\\否定:よくなかった\end{tabular} &\begin{tabular}{l}肯定:よかったです\\否定:\begin{tabular}{l}よくなかったです\\よくありませんでした\end{tabular}\end{tabular} \\
\end{tabularx}

形容词“いい”比较特殊,发生词尾变化或者接续“ない”的时候要变成“よい”


\subsubsection{形容词2}
形容词2由词干和词尾组成,有活用变化,词尾用“だ”表示
\paragraph{1.词典形}
词典形即为词干。
\paragraph{2.连体形}
词尾加上な,做连体修饰语

\begin{tabularx}{\textwidth}{l *{4}{>{\centering\arraybackslash}X}}
\rowcolor{grey!25}
\textbf{词典形}  & \textbf{词尾} & \textbf{敬体非过去肯定形式} & \textbf{例句}  \\
好き&な&好きな&ここは私の\textbf{好きな}町です\\
嫌い&な&嫌いな&子どもの時、あの人は\textbf{嫌いな}人でした\\
綺麗&な&綺麗な&\textbf{綺麗な}紅葉です\\
\end{tabularx}
\paragraph{3.连用形}
\subparagraph{第一连用形}
 出自第12课文法:将简体⾮过去时的肯定形式
中的词尾“”换成“”,在句中做连⽤修饰语,⽤来修饰⽤⾔
\begin{tabularx}{\textwidth}{l*{2}{>{\centering\arraybackslash}X}}
\rowcolor{grey!25}
\textbf{简体非过去时肯定形式}&\textbf{第一连用形} & \textbf{例句}\\
楽だ&楽に&どうぞ\textbf{楽に}してください\\
綺麗た&綺麗に&体を\textbf{きれいに}してから⼊るんです\\
静かだ&静かに&みなさん、\textbf{静かに}してください\\
\end{tabularx}

\subparagraph{第二连用形}
在句中表示并列、中顿。变化方式为词典形+“で”
\begin{tabularx}{\textwidth}{l *{4}{>{\centering\arraybackslash}X}}
\rowcolor{grey!25}
\textbf{敬体非过去肯定形式}  & \textbf{第二连用形} & \textbf{例句}  \\
鮮やかです&鮮やかで&この色は\textbf{鮮やかで}、綺麗です\\
賑やかです&賑やかで&ここは\textbf{賑やかで}、便利です\\
有名です&有名で&私の故郷は\textbf{有名で}、人が多いです\\
\end{tabularx}
\paragraph{4.终止形}

形容词2敬体非过去时肯定和否定形式:

词典形+“だ”构成形容词2终止形简体非过去肯定形式。

词典形+“です”构成形容词2终止形敬体非过去肯定形式。

各种活用都在“です”上面进行。

\begin{tabularx}{\textwidth}{l *{5}{>{\centering\arraybackslash}X}}
\rowcolor{grey!25}
\textbf{词典形} & \textbf{词干} & \textbf{词尾} & \textbf{敬体非过去肯定形式} & \textbf{敬体非过去否定形式}  \\
好き&好き&だ&好きです&\begin{tabular}{l}好きではないです\\好きではありません\end{tabular}\\
有名&ゆうめい&だ&有名です&\begin{tabular}{l}有名ではないです\\有名ではありません\end{tabular}\\
綺麗&きれい&だ&綺麗です&\begin{tabular}{l}綺麗ではないです\\綺麗ではありません\end{tabular}\\
\end{tabularx}

形容词2敬体非过去时肯定和否定形式:

词典形+“でした”构成形容词2终止形敬体过去肯定形式。

\begin{tabularx}{\textwidth}{l *{4}{>{\centering\arraybackslash}X}}
\rowcolor{grey!25}
\textbf{词典形} & \textbf{非过去肯定形式} & \textbf{过去肯定形式} & \textbf{非过去否定形式}  \\
好き&好きです&好きでした&\begin{tabular}{l}好きではなかったです\\好きではありませんでした\end{tabular}\\
有名&有名です&有名でした&\begin{tabular}{l}有名ではなかったです\\有名ではありませんでした\end{tabular}\\
綺麗&綺麗です&綺麗でした&\begin{tabular}{l}綺麗ではなかったです\\綺麗ではありませんでした\end{tabular}\\
\end{tabularx}

\subsubsection{补助形容词“ない”}
接在形容词1的第一连用形后面,表示对其性质
状态
变化等的否定。

这个词本身有活用变化,规律与形容词1大致相同,在句中可以做连体修饰语,也可以做简体结句。
\begin{tabularx}{\textwidth}{l *{3}{>{\centering\arraybackslash}X}}
\rowcolor{grey!25}
& \textbf{简体}  & \textbf{敬体} & \textbf{例句} \\
 非过去形式&ない&\begin{tabular}{l}ないです\\ありません\end{tabular} &\begin{tabular}{l}ここは東京ほど暑く\textbf{ない}\\ここは東京ほど暑く\textbf{ないです}\\ここは東京ほど暑く\textbf{ありません}\end{tabular} \\
 过去形式&なかった&\begin{tabular}{l}なかったです\\ありませんでした\end{tabular} &\begin{tabular}{l}昨日は涼し\textbf{くなかった}\\昨日は涼し\textbf{くなかったです}\\昨日は涼しく\textbf{ありませんでした}\end{tabular} \\
\end{tabularx}


\subsection{动词}
动词用于表示人或事物的行为、动作、状态、变化、作用、存在等,属于用言,有词尾活用变化。在句子中可以做谓语、连体修饰语等。

⽇语的动词有他动词和⾃动词之分。他动词相当于及物动词,可以带宾语,宾语⽤格助词“在”表⽰;⾃动词相当于不及物动词,不带宾語。

根据活⽤变化规律,
⼀般把⽇语动词分为三类。即:

动词1(也称1类动词,传统称五段活⽤动词)、动词2(也称2类动词,传统称⼀段活⽤动词)和动词3(也称3类动词,其中,“来”传统称カ变活⽤动词,“する”传统称变活⽤动词)。

动词1和动词2属于规则变化动词,动词3属于不规则变化动词。

动词词典形(也称基本形)有活⽤变化。各类动词的区别和特点如下:
\subsubsection{动词1}
动词1的特点是,词尾是⼀个假名,该假名在50⾳图的“う”段上;词尾前⾯的部分是词⼲。
\begin{tabularx}{\textwidth}{l *{3}{>{\centering\arraybackslash}X}}
\rowcolor{grey!25}
\textbf{词典形} & \textbf{词干} & \textbf{词尾}  \\
 撮る&撮&る\\
 行く&行&く\\
 遊ぶ&遊&ぶ\\
\end{tabularx}
\subsubsection{动词2}
动词2的特点是,词尾由两个假名组成,最后⼀个假名是“る”,“る”前⾯的假名在50⾳图的“い段”或“え段”上:词尾前⾯的部分是词⼲。

\begin{tabularx}{\textwidth}{l*{3}{>{\centering\arraybackslash}X}}
\rowcolor{grey!25}
\textbf{词典形}&\textbf{词干}&\textbf{词尾}\\
見る&見&見る\\
起きる&起&きる\\
食べる&食&べる\\
\end{tabularx}
\subsubsection{动词3}
动词3包括“サ变”和“カ变”动词两种类型。

サ変的尾“する”、“する”前⾯的部分是词⼲,⽽且词⼲⼀般是汉语性词汇。“する”除了做サ变动词词尾外,还可以作为⼀
个独⽴的サ变动词使⽤,这时它既是词⼲,又是词尾。
\begin{tabularx}{\textwidth}{l *{3}{>{\centering\arraybackslash}X}}
\rowcolor{grey!25}
\textbf{词典形} & \textbf{词干} & \textbf{词尾}  \\
 勉強する&勉強&する\\
 する&する&する\\
\end{tabularx}
此外,カ变动词只有“来”⼀个词,它既是词⼲,又是词尾
\subsubsection{动词终止形1}
做句尾的动词的所有词形都可以叫“终⽌形”。

动词的“终⽌形”有
敬体和简体之分:

\paragraph{⾮过去时肯定、否定形式}

敬体⾮过去时的肯定、否定形式是由词典形变化⽽来的,它⼀般⽤于表⽰未来的动作、状态、变化;也可以表⽰不受特定时间限制的超时的动作、变化等。各类动词的肯定、否定形式变化⽅法如下:

\subparagraph{动词1:}
动词1的敬体⾮过去时肯定形式(也称“ます形”)是把动词词典形的词尾由“う段”假名换成该⾏的“い段”假名,然后后续敬体助动词”ます”;否定形式是把“ます”換成“ません”。
\subparagraph{动词2:}
动词2的敬体⾮过去时肯定形式(也称“ます形”)是把词典形的词尾最后⼀假名“る”換成“ます”;否定形式是“ます”換成“ません”。
\subparagraph{动词3:}

サ变活⽤动词的敬体⾮过去时肯定形式(也称“ます形”)是把词尾“する”換成“し”、然后再加上“ます”;否定形式是把“ます”換成“ません”。

カ变活⽤动词的敬体⾮过去时肯定形式(也称“ます形”)是把“来る”換成“来”、然后再加上“ます”:否定形式是把“ます”換成“ません”。

\paragraph{过去时肯定、否定形式}
敬体过去时的肯定、否定形式是由词典形变化⽽来的,⼀般表⽰动作、状态、变化存在于说话的时点之前。变化的⽅法是:

把⾮去肯定形式中的“ます”換成“ました”。⼀般可以理解为“~了”。其否定形式是把“ました”換成“ませんでした”。⼀般可以“过去没~”。以上讲到的⼏种动词的变化⽅法如下表:
\begin{tabularx}{\textwidth}{l *{6}{>{\centering\arraybackslash}X}}
\rowcolor{grey!25}
\textbf{词典形}&\textbf{动词类型}&\textbf{非过去时的肯定形式}&\textbf{非过去时的否定形式}&\textbf{过去时的肯定形式}&\textbf{过去时的否定形式}\\
行く&动词1&行います&行いません&行いました&行いませんでした\\
遊ぶ&动词1&遊びます&遊びません&遊びました&遊びませんでした\\
撮る&动词1&撮ります&撮りません&撮りました&撮りませんでした\\
 見る&动 词 2&見ます&見ません&見ました&見ませんでした\\
 起きる&动 词 2&起きます&起きません&起きました&起きませんでした\\
 食べる&动 词 2&食べます&食べません&食べました&食べませんでした\\
 利用する&动 词 3&利用します&利用しません&利用しました&利用しませんでした\\
 する&动 词 3&します&しません&しました&しませんでした\\
 来る&动 词 3&来ます&来ません&来ました&来ませんでした\\
\end{tabularx}

\subsubsection{动词终止形2}

\paragraph{简体⾮过去时肯定形式}
简体⾮过去时肯定形式,即词典形(也称基本形、简体),有词尾活
⽤变化,可以做连体修饰语或⽤来结句等。

\paragraph{简体⾮过去时否定形式}
动词经活⽤变化,后续补助形容词“”,构成简体⾮过去时的否
定形式,称作“ない形”。

⼀般可以“不~”。“ない”接续动词的⽅法如下:

\subparagraph{动词1}
将词尾的ウ段假名換成⾏的ア段假名,加“ない”。以“う”为詞尾的需把“う”換成“わ”、再加“ない”。
\subparagraph{动词2}
将词尾最后⼀个假名“る”去掉,加“ない”。
\subparagraph{动词3}
サ変动词是将词尾“する”換成”し”加“ない”;カ変动词是将“来る”換成“来”加“ない”。具体変化⽅法如下表:
\begin{tabularx}{\textwidth}{l *{4}{>{\centering\arraybackslash}X}}
\rowcolor{grey!25}
\textbf{动词类型} &\textbf{词典形与变化}&\textbf{非过去时否定形式}&\\
动词1&
\begin{tabular}{l}買う->買わ\\書く->書か\\済む->済ま\\集まる->集まら\\見る->見\end{tabular} &
\begin{tabular}{l}買\textbf{わない}\\書\textbf{かない}\\済\textbf{まない}\\集\textbf{まらない}\\見\textbf{ない}\end{tabular}\\ \hline
动词2&
\begin{tabular}{l}起きる->起き\\教える->教え\\勉強する->勉強\end{tabular} &
\begin{tabular}{l}起き\textbf{ない}\\教え\textbf{ない}\\勉強\textbf{ない}\end{tabular}\\ \hline
动词3&
\begin{tabular}{l}する->し\\来る->来\end{tabular} &
\begin{tabular}{l}\textbf{しない}\\来\textbf{ない}\end{tabular}\\ \hline

\end{tabularx}
\subsubsection{动词终止形3}
\subsubsection{动词连用形1}
动词的第⼀连⽤形(也称“主形”)是把敬体⾮过去时的肯定形式中的“ます”去掉。它在句⼦中表⽰中頓、并列。如下表:

\begin{tabularx}{\textwidth}{l *{4}{>{\centering\arraybackslash}X}}
\rowcolor{grey!25}
\textbf{动词类型} &\textbf{词典形}&\textbf{敬体非过去时肯定形式}&\textbf{第一连用形}&\\
动词1&
\begin{tabular}{l}もらう\\選べ\end{tabular} &
\begin{tabular}{l}もらいます\\選びます\end{tabular}&
\begin{tabular}{l}もらい\\選び\end{tabular}\\ \hline
动词2&
\begin{tabular}{l}起きる\\出かける\end{tabular} &
\begin{tabular}{l}起きます\\出かけます\end{tabular}&
\begin{tabular}{l}起き\\出かけ\end{tabular}\\ \hline
动词3&
\begin{tabular}{l}利用する\\する\\来る\end{tabular} &
\begin{tabular}{l}利用します\\します\\来ます\end{tabular}&
\begin{tabular}{l}利用し\\し\\来\end{tabular}\\ \hline

\end{tabularx}
\begin{itemize}
    \item 去年の誕⽣⽇に母からセーターを\textbf{もらい}、⽗から⾃転⾞をもらいました
    \item わたしは友達と嵐⼭へ\textbf{⾏き}、写真をたくさん撮りました
\end{itemize}
\subsubsection{动词连用形2}
动词词典形后续“て”,构成了第⼆连⽤形(也称“て形”),它常常⽤“Vて”来表⽰。“て”与动词接续的⽅法如下:
\paragraph{动词1}
\subparagraph{当词尾为“く”“ぐ”时}
发⽣“い⾳変”。即:先把词尾“く”或“ぐ”去掉、換成“い”、加“て”。当词尾为“ぐ”时,”要変成”で”。
\subparagraph{当词尾为“う”“つ”“る”时}
发⽣“促⾳変”。即:先把词尾“う”“つ”“る”去掉,換成促⾳符号“っ”,加上“て”。
\subparagraph{当词尾为“ぬ”“ぶ”“む”时}
发⽣“拨⾳変”。即:先把词尾“ぬ”“ぶ”“む”去掉,換成”ん”、加上”で”。
\subparagraph{当词尾为“す”时}
先把“す”去掉,換成”し”、加上て
\paragraph{动词2}
把最后⼀个假名“る”去掉,換成“て”。
\paragraph{动词3}
サ変是把司尾“する”去掉,換成“し”、加“て”;
カ変动词是把“来(く)る”変成“来(き)”、加“て”

\begin{tabularx}{\textwidth}{l *{6}{>{\centering\arraybackslash}X}}
\rowcolor{grey!25}
\textbf{動詞1}&\textbf{第二連用形}&\textbf{動詞2}&\textbf{第二連用形}&\textbf{動詞3}&\textbf{第二連用形}\\
書く&書いて&見る&見て&する&して\\
脱ぐ&脱いで&起きる&起きて&勉強する&勉強して\\
買う&買って&教える&教えて&来る&来て\\
待つ&待って&&&&\\
降る&降って&&&&\\
死ぬ&死んで&&&&\\
呼ぶ&呼んで&&&&\\
読む&読んで&&&&\\
話す&話して&&&&\\
\end{tabularx}
\subsubsection{动作的持续体}
”ている”接在动词后⾯,与”て”的接续⽅法相同。它本⾝的活⽤変化按动词2活⽤変化规则进⾏,敬体形式”ています”。

“ている”表⽰说话时点上正在进⾏的动作或持续的状态。⼀般可以译为“正在~”“在~呢”。例如:
\begin{itemize}
    \item みんなで⽔ギョーザを作っ\textbf{ています}。(持续的动作)
    \item ⾺さんは今、本を読ん\textbf{でいます}。(持续的动作)
    \item すぐわたしの発表になりますので、とても緊張し\textbf{ています}。(持续的状态)
    \item ⽊村さんは飯⽥橋に住ん\textbf{でいます}。(持续的状态)
\end{itemize}
例句中的“作る”“読む”表⽰动作的持续、后续“ています”表⽰“説活点上正在⾏的动作”:“緊張する”“住む”是瞬间可以完成的动作,它们后续“ています”表⽰“动作完成后持续的状态”。

”ている”的否定形式是“ていません”、前経常与副词“まだ”搭配使⽤,表⽰这⼀动作或状态尚未进⾏或出现。例如:

\begin{itemize}
    \item わたしはまだ昼⾷を取っ\textbf{ていません}。
    \item そのことはまだ先⽣に話し\textbf{ていません}。
    \item 今⽇の新聞はまだ読ん\textbf{でいません}。
\end{itemize}
\subsubsection{接受动词}
\paragraph{もらう}
⽤于⾃⼰或⾃⼰⼀⽅的⼈从别⼈那⾥得到,有得恩惠的含义。受益者在句中做主语,但主语是第⼀⼈称时可以省略:施益者⼀般⽤“に”或“から”表⽰、做补语。“もらう”多⽤同龄,同辈⼈。

⼀般译为“从~得到~”。有时虽是对⽅主动给⾃⼰或⾃⼰⼀⽅的⼈的,但为了表⽰感激之情,也⽤这种说法。这时可以译为“~给(我)~”。
\begin{itemize}
    \item わたし\textbf{は}同級⽣\textbf{から}プレゼントを\textbf{もらいました}。
    \item ⽗\textbf{から}⾃転⾞を\textbf{もらいました}。
    \item ⽇本⼈の友達\textbf{に}辞書を\textbf{もらいました}。
\end{itemize}
\paragraph{くれる}
⽤于别⼈给说话⼈或属于说话⼈⼀⽅的⼈什么东西等,说话⼈或属于说话⼈⼀⽅的⼈是受益者。施益者在句中做主语;

受益者⽤“に”表⽰,做补语。当受益者第⼀⼈称肘被省略。“くれる”
多⽤于同龄、同辈⼈或⽐较亲近的长辈。
⼀般译为“给我~”。
\begin{itemize}
    \item 母\textbf{は}(わたし\textbf{に})プレゼントを\textbf{くれ}ました。
    \item ⽗\textbf{は}(僕\textbf{に})⾃転⾞を\textbf{くれ}ました。
    \item 佐藤さん\textbf{は}妹\textbf{に}ボールペンを\textbf{くれ}ました。
\end{itemize}
\paragraph{あげる}
⽤于说话⼈或属于说话⼈⼀⽅的⼈给别⼈什么东西等时。

说话⼈或属于说话⼈⼀⽅的⼈在句中做主语;受益者后⾯⽤“に”表⽰,
做补语。多⽤于同龄、同辈⼈或⽐较亲近的长辈。
⼀般译为“给(对⽅或第三者)~”。
\begin{itemize}
    \item 僕\textbf{は}⽗\textbf{に}セーターを\textbf{あげ}ます。
    \item 弟\textbf{は}同級⽣\textbf{に}漫画を\textbf{あげ}ました。
    \item ⽊村さん\textbf{は}⾺さん\textbf{に}⼿作りの誕⽣⽇祝いを\textbf{あげ}ました。
\end{itemize}
\subsubsection{动作的持续体}
\paragraph{もらう}
\paragraph{くれる}
\paragraph{あける}

\subsubsection{助动词}
\paragraph{1.た}
\paragraph{2.たい}

\subsection{接续词}
接续词相当于汉语的连词,没有活用变化,是独立的词语。用于连接两个及以上的词、句子等,表示前后两个句子的接续关系。有时也直接用于句首。
\paragraph{1.でも}
表转折,多用于口语,意为“不过、可是、但是”
\begin{itemize}
    \item 北京の夏も暑いです。\textbf{でも}、東京ほど蒸し暑いアィません
    \item 北京の冬は寒いです。\textbf{でも}、去年は暖冬でした
\end{itemize}
\paragraph{2.ところで}

用于句首,表示转换话题。一般译为“不过”、“可是”等。

\begin{itemize}
    \item たくさん建物がありますね。\textbf{ところで}、どこが教育学部ですか。
    \item この大学は駅に近いですね。\textbf{ところで}、学生はどのぐらいいますか。
    \item 池に鯉がたくさんいますね。\textbf{ところで}、大学の本部はどこにありますか。
\end{itemize}
\paragraph{3.それに}
⽤于连接两个句⼦或词组,表⽰累加。可以译为“⽽且”
“再加上”。
\begin{itemize}
    \item いよいよ20歳になります。\textbf{それに}、⽇本で初めての誕⽣⽇です。
    \item ここは駅に近いです。\textbf{それに}、⽣活も便利です。
    \item ボールペン、\textbf{それに}、本をください。
\end{itemize}
\paragraph{4.だから}
⽤于连接两个句⼦,表⽰因果关系,多⽤于説話⼈的主規判断。可以“因为~所以”。
\begin{itemize}
    \item 昨⽇は⾬が降りました。だ\textbf{}から、どこへも⾏きませんでした。
    \item あした、友達の誕⽣⽇です。\textbf{だから}、プレゼントを買いに⾏きます。
    \item 天気予報は聞きませんでした。\textbf{だから}、あしたの天気は分かりません。
\end{itemize}
\paragraph{5.ですから}
⽤于接两个句⼦,表⽰因果关系、是“だから”的郑重説法。⼀般可以“因~所以~”。
\begin{itemize}
    \item このお湯は家族全員が使います。\textbf{ですから}、体をきれいにしてから⼊るんです。
    \item 昨⽇は⾬が降りました。\textbf{ですから}、どこへも⾏きませんでした。
    \item あしたは母の誕⽣⽇です。\textbf{ですから}、⽗と⼀緒にプレゼントを買いに⾏きます。
\end{itemize}
\paragraph{6.それから}
⽤来连接两个或两个以上的句⼦,表⽰顺序,有时也有补充、附加之意。⼀般意为“然后~”“有~”。
\begin{itemize}
    \item まず、外で体を流して、\textbf{それから}湯船に⼊ってください。
    \item ⾺さん,佐藤さん、\textbf{それから}⽊村さんは、今教室にいます。
    \item 今⽇は教育学部、⼈⽂学部,⼯学部\textbf{それから}⽣協へも⾏きました。
\end{itemize}
\paragraph{7.では(じゃ)}
⽤于句⾸,表⽰转换话题,意为“那么”。“では”既可⽤于书⾯也可以⽤于口语,使⽤范围⽐较⼴泛:“じゃ”⼀般⽤于口语,使⽤对象多是平辈⼈、晚辈⼈。
\begin{itemize}
    \item \textbf{では}、みんなで教室をきれいにしましょう。
    \item \textbf{じゃ}、今⽇はお母さんの⼿料理にします。
    \item \textbf{では}、いただきます
\end{itemize}
\paragraph{8.つまり}
⽤于后⼀个分句或词组的句⾸,表⽰对前⼀个分句或词组的说明、
补充或归纳。可以译为“即~”“也就是说~”。例如:
\begin{itemize}
    \item ⽔ギョーザはわたしの「外メニュー」です。\textbf{つまり}、わたしはよく⽔ギョーザで友達を招待します。
    \item いとこ、\textbf{つまり}⽗のおとうとの息⼦で、今⼤学の3年⽣です
\end{itemize}
\paragraph{9.しかし}

\subsection{感叹词}
\paragraph{1.はい}
回答不带疑问词的疑问句,后面常接そうです,表示肯定

\paragraph{2.いいえ}
回答不带疑问词的疑问句,后面常接そうではありません,表示否定


\subsection{后缀}
\paragraph{1.さん}
\paragraph{2.ごろ}
接在与时间、⽇期相关的体词后,表⽰⼤概的时间、⽇期。为了加
强語気、”ごろ”后⾯可以加”に”。
⼀般意为“⼤約”“前后”“左右”。
\begin{itemize}
    \item 朝9時半\textbf{ごろ}電⾞に乗りました。
    \item 毎⽇8時\textbf{ごろ}に起きます。
    \item それは12⽉\textbf{ごろ}のことでした。
\end{itemize}

\end{document}